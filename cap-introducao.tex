%% ------------------------------------------------------------------------- %%
\chapter{Introdução}
\label{cap:introducao}

Em 1936, Alan Turing publicou um artigo chamado \textit{On Computable Numbers, with an application to the Entscheidungsproblem} (\citeyear{turing1936a}) no qual descreveu uma máquina teórica capaz de resolver qualquer problema que pudesse ser descrito através de instruções pré-definidas numa fita de papel. Este artigo tornaria-se a base para a computação e invenção dos computadores. Durante a segunda guerra, Turing construiu uma máquina capaz de quebrar a maioria do código Morse secreto usado nas comunicações pela força naval alemã.  Segundo alguns especialistas, a guerra teria durado mais tempo se não fosse a máquina inventada por Alan Turing para quebrar a criptografia das mensagens. A humanidade deu um grande salto graças ao avanço dos computadores desde o fim da segunda guerra até os dias de hoje. Este salto tecnológico permitiu o avanço em diversas áreas como saúde, telecomunicações, aviação, engenharia, financeira etc. 

Com o aumento do poder de processamento computacional dobrando a cada 18 meses, Lei de Moore, novos avanços em tecnologia surgem como a Internet da Coisas (IoT), cidades inteligentes (Smart Cities), Indústria 4.0. Todas tem um ponto em comum: computação úbiqua, uso de inteligência artificial e big data. E um outro ponto é a automação. Governos em todo o mundo estão preocupados com o advento da inteligência artificial e automação, pois muitos especialistas dizem que haverá um desemprego em massa. A maioria dos empregos que tem uma tarefa repetitiva, que não exige criatividade e interação social, serão automatizados. Em sua maioria, são empregos exercidos por pessoas com baixa qualificação \citep{europarl:2016}. Segundo um relatório da OCDE, este desemprego em massa não ocorrerá tão rapidamente, pois o computador consegue fazer somente tarefas bem definidas e a maioria dos trabalhos não tem uma definição clara das tarefas a serem feitas \citep{ArntzGregoryZierahn16oecd}.

Em um item todos os governos concordam: há uma necessidade em preparar a nova geração para esta nova era. Hoje há um defícit de programadores em todo o mundo. E com o advento do IoT, Indústria 4.0, a demanda por desenvolvedores e programadores irá aumentar ainda mais. Governos de todo o mundo estão empenhados em um projeto de ensino de programação no currículo básico das escolas para atender a esta demanda. Alan Perlis disse em 1960 que programação deveria ser ensinada a todos. O modo de pensar "algoritmicamente" é útil para diversos problemas em nosso cotidiano. E com a popularização e a presença em massa dos computadores em nosso cotidiano, a importância de saber programar aumenta. Governos e empresas multinacionais juntaram forças para ensinar programação à populacao através de iniciativas como o code.org, udacity, scratch etc.

Mas aprender a programar não é uma tarefa fácil. Diversos estudos que apontam para a alta taxa de evasão dos alunos dos cursos de Ciência da Computação \citep{Watson:2014:FRI:2591708.2591749}. E outros estudos apontam a alta taxa de reprovação por parte dos alunos nos cursos de introdução à programação \citep{Watson:2014:FRI:2591708.2591749, bosse:2015:cbie}. Segundo um estudo preliminar feito por \cite{bosse:2015:cbie}, a taxa de reprovação nos cursos introdutórios de programação ultrapassam a taxa de 30\% somente na Universidade de São Paulo. A reprovação no curso de introdução à programação é um dos fatores que contribuem para a evasão do curso. Segundo os dados do censo do Ensino Superior divulgado pelo MEC em 2016, a taxa de evasão média para cursos de bacheralado é em torno de 22\%, enquanto para os cursos de bacharelado em Computação, a taxa de evasão supera 28\%.

Segundo \cite{Kalelioglu:2016}, o pensamento computacional é um processo para resolução de problemas. Inicia-se com a identificação do problema através da abstração e decomposição. Para entender melhor e solucionar o problema, é necessário coletar, representar e analisar os dados. No processo de coleta e análise, reconhecimento de padrões, conceptualização e representação dos dados devem ser levados em consideração. Para obter soluções mais precisas, processos cognitivos como lógica matemática, algortimos podem ser empregados. Para implementar a solução, modelagem, automação e simulação devem ser utilizadas para verificar a eficácia da solução. Por último, aplicar a solução obtida para diferentes tipos de problemas deve ser realizado para generalizar a solução. Segundo \cite{Kalelioglu:2016} o pensamento computacional envolve diversos modelos mentais como abstração, resolução de problemas, reconhecimento de padrões, decomposição, generalização etc.

Além da necessidade de desenvolver o pensamento computacional, é importante saber quais são as principais dificuldades encontradas pelos alunos. Segundo \cite{jenkins:2002}, há diversos fatores que contribuem para a dificuldade do aprendizado. Programar exige múltiplas habilidades, é uma tarefa de vários processos, é novidade para maioria dos alunos ingressantes, depende da motivação do aluno, interese, ritmo da aula etc.

Há estudos que tentam identificar quais são os erros mais comuns cometidos pelos alunos na aprendizagem de programação \citep{Hristova:2003:ICJ:792548.611956, Caceffo:2016:DCS:2839509.2844559}. A partir dos erros comumente cometidos pelos alunos, é possível elaborar materiais e alterar o ritmo da aula para um melhor aprendizado do estudante. Porém, a maioria dos estudos adotam a abordagem de entrevista com alunos e professores, catalogação dos erros verificados e posterior análise para validá-los.

Com o aumento do poder computacional, crescente uso de ferramentas de aprendizagem de máquina e a grande oferta de repositórios de código-fontes como Github, por exemplo, nasce o termo Big Code \citep{DBLP:journals/corr/BhatiaS16}. Big Code refere-se ao conhecimento que podemos adquirir a partir da análise dos repositórios de código fonte através de ferramentas de aprendizagem de máquina. Este é um novo desafio, construir ferramentas que sejam capazes de aprender a partir dos repositórios de código fonte. Com o conhecimento que podemos adquirir a partir da análise dos repositórios, podemos construir ferramentas para analisar a arquitetura de um projeto, verificar se o código fonte ou o nome do método está de acordo com a convenção, detectar erros de execução de forma antecipada.

Com as ferramentas de aprendizagem de máquina, é possível também identificar quais são os erros mais comuns cometidos pelos alunos. Esta técnica é utilizada para dar uma resposta automática para o aluno, permitindo que ele receba uma avaliação mais rápida que a dada pelo professor ou monitor. Segundo um estudo feito por \cite{DBLP:journals/corr/BhatiaS16}, o uso de aprendizagem de máquina permitiu criar uma ferramenta para detectar erros de sintaxe e corrigi-los. Um texto publicado no site da Microsoft, mostrou como o uso do algoritmo de \textit{Deep Learning} foi eficaz para identificar as correções necessárias no código fonte submetido pelo aluno.

As perguntas que esta dissertação tentará responder são quais padrões de erros são identificáveis e mais comuns que uma ferramenta de aprendizagem de máquina consegue detectar a partir de um repositório de código-fonte previamente analisados? Quais são os tipos de erros facilmente identificáveis? Qual a diferença entre os padrões de erros encontrados pela ferramenta e os padrões de erros identificados na literatura, principalmente, padrões de erros obtidos através de entrevistas com alunos e professores? Para cada padrão de erro, qual são as propostas de soluções?


\section{Considerações Preliminares}
\label{sec:consideracoes_preliminares}

O uso de algoritmo de aprendizagem para correção automática de exercícios não é novidade. Porém, boa parte dos estudos foram feitos para plataformas MOOC, no qual a maioria dos exercícios são avaliados através de uma base de testes e a nota representa a porcentagem de testes que o exercício passou. Neste estudo, identificaremos os padrões de erros através do uso do algoritmo de redes neurais aplicado num conjunto de exercícios previamente avaliados por monitores e professores. Avaliação feita por monitores e instrutores levam em consideração se o aluno entendeu conceitos como instruções de repetição, instruções de decisão, chamada de métodos, recursão, uso de vetores, ponteiros, passagem de parâmetros etc. Este tipo de avaliação é diferente das avaliações baseadas em porcentagem de testes.

%% ------------------------------------------------------------------------- %%
\section{Objetivos}
\label{sec:objetivo}

O objetivo da pesquisa é detectar padrões de erros em código-fonte de alunos previamente analisados por monitores e instrutores através do uso de redes neurais. A hipótese inicial é que os alunos com notas menores cometem os mesmos padrões de erros, diferente dos alunos que tiveram notas mais altas. Outra hipótese é que estes padrões de erros são facilmente identificáveis e comuns como condição de parada errada na instrução de repetição, uso incorreto de passagem de parâmetros nos métodos, condição errada na instrução de decisão etc.

%% ------------------------------------------------------------------------- %%
\section{Contribuições}
\label{sec:contribucoes}



%% ------------------------------------------------------------------------- %%
\section{Organização do Trabalho}
\label{sec:organizacao_trabalho}

No Capítulo~\ref{cap:conceitos}, apresentamos os conceitos ... Finalmente, no
Capítulo~\ref{cap:conclusoes} discutimos algumas conclusões obtidas neste
trabalho. Analisamos as vantagens e desvantagens do método proposto ... 

As sequências testadas no trabalho estão disponíveis no Apêndice \ref{ape:sequencias}.
