%% ------------------------------------------------------------------------- %%
\chapter{Introdução}
\label{cap:introducao}

Aprender a programar não é uma tarefa fácil. Há diversos estudos que apontam para a alta taxa de evasão dos alunos dos cursos de Ciência da Computação ???. E outros estudos apontam a alta taxa de reprovação por parte dos alunos nos cursos de introdução à programação. A reprovação no curso de introdução à programação é um dos fatores que contribuem para a desistência do curso (preciso achar algum estudo que aponte isso). 

Para aprender a programar, segundo XXX é necessário que o aluno tenha as seguintes habilidades ????? (reescrever). Estas habilidades também não são garantias de aprendizagem, há diversos outros fatores a serem considerados como social, psicológico, motivação, contrato didático entre professor e aluno, didática do professor etc. 

E segundo o estudo YYY, os instrutores e professores tem dificuldade em identificar quais são os erros mais cometidos pelos alunos. De acordo com o YYY, a maioria dos professores não conseguiu identificar quais são as dificuldades dos alunos, quando confrontados com entrevistas feitas com ambos e após uma análise (reescrever).

Há estudos que tentam identificar quais são os erros mais comuns cometidos pelos alunos na aprendizagem de programação. Porém, a maioria dos estudos adotam a abordagem de entrevista com alunos e professores, catalogação dos erros verificados e posterior análise para validá-los (verificar se está certo??) \cite{Hristova:2003:ICJ:792548.611956}.  (dificuldade e erro sao a mesma coisa?)

Com o aumento do poder computacional, crescente uso de ferramentas de aprendizagem de máquina e a grande oferta de repositórios de código-fontes como Github, por exemplo, nasce o termo Big Code. Big Code refere-se ao conhecimento que podemos adquirir a partir da análise dos repositórios de código fonte através de ferramentas de aprendizagem de máquina. Este é um novo desafio, construir ferramentas que sejam capazes de aprender a partir dos repositórios de código fonte. Com o conhecimento que podemos adquirir a partir da análise dos repositórios, podemos construir ferramentas para analisar a arquitetura de um projeto, verificar se o código fonte ou o nome do método está de acordo com a convenção, detectar erros de execução de forma antecipada. (reescrever com referências)

Com as ferramentas de aprendizagem de máquina, é possível também identificar quais são os erros mais comuns cometidos pelos alunos. Esta técnica é utilizada para dar uma resposta automática para o aluno, permitindo que ele receba uma avaliação mais rápida que a dada pelo professor. Segundo o estudo XXX, o uso de aprendizagem de máquina permitiu criar uma ferramenta para detectar erros de sintaxe corrigi-los. Um texto publicado no site da Microsoft, mostrou como o uso do algoritmo de \textit{Deep Learning} foi eficaz para identificar as correções necessárias que o aluno deveria fazer para que o programa estivesse correto (reescrever).

A pergunta que esta dissertação irá responder é quais erros são identificáveis e mais comuns que uma ferramenta de aprendizagem de máquina consegue detectar a partir de um repositório de código-fonte previamente analisados? Quais são os tipos de erros facilmente identificáveis? Qual a diferença entre os erros encontrados pela ferramenta e os erros identificados na literatura, principalmente, erros obtidos através de entrevistas com alunos e professores? Esta técnica é aplicável a todas as linguagens? Caso seja aplicável em todas as linguagens, quais erros são mais facilmente identificáveis?


\section{Considerações Preliminares}
\label{sec:consideracoes_preliminares}



 

%% ------------------------------------------------------------------------- %%
\section{Objetivos}
\label{sec:objetivo}

Texto texto texto texto texto texto texto texto texto texto texto texto texto
texto texto texto texto texto texto texto texto texto texto texto texto texto
texto texto texto texto texto texto.

%% ------------------------------------------------------------------------- %%
\section{Contribuições}
\label{sec:contribucoes}



%% ------------------------------------------------------------------------- %%
\section{Organização do Trabalho}
\label{sec:organizacao_trabalho}

No Capítulo~\ref{cap:conceitos}, apresentamos os conceitos ... Finalmente, no
Capítulo~\ref{cap:conclusoes} discutimos algumas conclusões obtidas neste
trabalho. Analisamos as vantagens e desvantagens do método proposto ... 

As sequências testadas no trabalho estão disponíveis no Apêndice \ref{ape:sequencias}.
