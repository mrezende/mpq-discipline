%% ------------------------------------------------------------------------- %%
\chapter{Conceitos}
\label{cap:conceitos}

Texto texto texto texto texto texto texto texto texto texto texto texto texto
texto texto texto texto texto texto texto texto texto texto texto texto texto
texto texto texto texto texto texto texto texto texto texto texto texto texto
texto texto texto texto texto texto texto texto texto texto texto texto texto
texto texto texto texto texto texto.

%% ------------------------------------------------------------------------- %%
\section{Fundamentos}\index{Área do trabalho!fundamentos}
\label{sec:fundamentos}

Texto texto texto texto texto texto texto texto texto texto texto texto texto
texto texto texto texto texto texto texto texto texto texto texto texto texto
texto texto texto texto texto texto texto texto texto texto texto texto texto
texto texto texto texto texto texto texto texto texto texto texto texto texto
texto texto texto.




%% ------------------------------------------------------------------------- %%
\section{Exemplo de Código-Fonte em Java}
\label{sec:exemplo_codigo_fonte}
Texto texto texto texto texto texto texto texto texto texto texto texto texto
texto texto texto texto texto texto texto texto texto texto texto texto texto
texto texto texto texto texto texto texto texto texto texto texto texto texto
texto texto texto texto texto texto texto.

% Foi utilizado o pacote listing para formatar c?igo fonte
% http://ctan.org/tex-archive/macros/latex/contrib/listings/listings.pdf
% Veja no preambulo do arquivo tese-exemplo.tex os par?etros de configura?o.

\begin{lstlisting}[frame=trbl]
    for(i = 0; i < 20; i++)
    {
        // Comentário 
        System.out.println("Mensagem...");
    }
\end{lstlisting}


%% ------------------------------------------------------------------------- %%
\section{Algumas Referências}
\label{sec:algumas_referencias}

É muito recomendável a utilização de arquivos \emph{bibtex} para o gerenciamento
de referências a trabalhos. Nesse sentido existem três plataformas gratuitas
que permitem a busca de referêcias acadêmicas em formato bib: 
\begin{itemize}
	\item \emph{CiteULike} (patrocinados por Springer): \url{www.citeulike.org}
	\item Coleção de bibliografia em Ciência da Computação: \url{liinwww.ira.uka.de/bibliography}
	\item Google acadêmico (habilitar bibtex nas preferências): \url{scholar.google.com.br}
\end{itemize}
Lamentavelmente, ainda não existe um mecanismo de verificação ou validação das
informações nessas plataformas. Portanto, é fortemente sugerido validar todas
as informações de tal forma que as entradas bib estejam corretas.  Também, tome
muito cuidado na padronização das referências bibliográficas: ou considere TODOS
os nomes dos autores por extenso, ou TODOS os nomes dos autores abreviados.
Evite misturas inapropriadas.



