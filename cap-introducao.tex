%% ------------------------------------------------------------------------- %%
\chapter{Introdução}
\label{cap:introducao}

O uso do primeiro computador na segunda guerra mundial por Alan Turing, foi um marco em nossa história. Segundo alguns especialistas, a guerra teria durado mais tempo se não fosse o computador inventado por Alan Turing para quebrar a criptografia das mensagens de comunicação da aviacao alemã. A humanidade deu um grande salto graças ao avanço dos computadores desde 1945 até os dias de hoje. Este avanço tecnológico permitiu a criação de aparelhos de diagnósticos médicos mais precisos como ressonância magnética computadorizada, ultrassom, tomorizacao computadorizada do cérebro, sequenciamento do genoma humano, telecóspios mais potentes e análise mais precisa dos dados etc. Com o aumento do poder de processamento computacional aumentando a cada 2 anos, Lei de Moore (verificar), alguns especialistas dizem que a humanidade irá entrar numa nova era: a era da máquina. 

Esta era coincide com outros eventos definidos por especialistas como o advento da Internet da Coisas (IoT), cidades inteligentes (Smart Cities), Indústria 4.0. Todas tem um ponto em comum: computação úbiqua, uso de inteligência artificial e big data. E um outro ponto é a automação. Governos em todo o mundo estão preocupados com o advento da inteligência artificial e automação, pois muitos especialistas dizem que haverá um desemprego em massa. A maioria dos empregos que tem uma tarefa repetitiva, que não exige criatividade e interacao social, serão automatizados. Em sua maioria, são empregos exercidos por pessoas com baixa qualificação (verificar). segundo um relatório da OCDE, este desemprego em massa não ocorrerá tão rapidamente, pois o computador consegue fazer somente tarefas bem definidas e a maioria dos trabalhos não tem uma definição clara das tarefas a serem feitas. 

Em um ponto todos os governos concordam: há uma necessidade em preparar a nova geração para esta nova era. Governos de todo o mundo estão empenhados em um projeto de ensino de programação no currículo básico das escolas. Como dizia XXX em 1970, aprender a ler é importante, porém com o advento dos computadores, é fundamental que as pessoas saibam programar. Há uma enorme demanda por programadores em todo o mundo, os salários dos programadores em lugares como os EUA, sao um dos mais altos, devido a escassez de mão de obra qualificada. Governos e empresas multinacionais juntaram forças para ensinar programação à populacao através de iniciativas como o code.org, udacity, scratch etc.

Mas aprender a programar não é uma tarefa fácil. Há diversos estudos que apontam para a alta taxa de evasão dos alunos dos cursos de Ciência da Computação \citep{Watson:2014:FRI:2591708.2591749} (acrescentar referencia do censo da educacao). E outros estudos apontam a alta taxa de reprovação por parte dos alunos nos cursos de introdução à programação \citep{Watson:2014:FRI:2591708.2591749, bosse:2015:cbie} (acrescentar referencia do censo da educacao). Segundo um estudo preliminar feito por \cite{bosse:2015:cbie}, a taxa de reprovação nos cursos introdutórios de programação ultrapassam a taxa de 30\% somente na Universidade de São Paulo. A reprovação no curso de introdução à programação é um dos fatores que contribuem para a evasão do curso. Segundo os dados do censo do Ensino Superior divulgado pelo MEC em 2016, a taxa de evasão média para cursos de bacheralado é em torno de 22\%, enquanto para os cursos de bacharelado em Computação, a taxa de evasão supera 28\% (citar referência quando o site do INEP voltar ao ar) (dados da Yorah Bosse, atualizar: Dados mais recentes de um relatório técnico que analisa os Censos da Educação Superior do Ministério da Educação (MEC), realizados pelo INEP/MEC entre      e      [  ], confirmam o problema: o que se observa entre os concluintes em cursos de computação é um abandono de cerca de   50\% dos alunos matriculados.).

Para aprender a programar, segundo XXX é necessário que o aluno tenha as seguintes habilidades ????? (reescrever). Estas habilidades também não são garantias de aprendizagem, há diversos outros fatores a serem considerados como social, psicológico, motivação, contrato didático entre professor e aluno, didática do professor etc. 

E segundo o estudo YYY, os instrutores e professores tem dificuldade em identificar quais são os erros mais cometidos pelos alunos. De acordo com o YYY, a maioria dos professores não conseguiu identificar quais são as dificuldades dos alunos, quando confrontados com entrevistas feitas com ambos e após uma análise (reescrever).

Há estudos que tentam identificar quais são os erros mais comuns cometidos pelos alunos na aprendizagem de programação. Porém, a maioria dos estudos adotam a abordagem de entrevista com alunos e professores, catalogação dos erros verificados e posterior análise para validá-los (verificar se está certo??) \cite{Hristova:2003:ICJ:792548.611956}.  (dificuldade e erro sao a mesma coisa?)

Com o aumento do poder computacional, crescente uso de ferramentas de aprendizagem de máquina e a grande oferta de repositórios de código-fontes como Github, por exemplo, nasce o termo Big Code. Big Code refere-se ao conhecimento que podemos adquirir a partir da análise dos repositórios de código fonte através de ferramentas de aprendizagem de máquina. Este é um novo desafio, construir ferramentas que sejam capazes de aprender a partir dos repositórios de código fonte. Com o conhecimento que podemos adquirir a partir da análise dos repositórios, podemos construir ferramentas para analisar a arquitetura de um projeto, verificar se o código fonte ou o nome do método está de acordo com a convenção, detectar erros de execução de forma antecipada. (reescrever com referências)

Com as ferramentas de aprendizagem de máquina, é possível também identificar quais são os erros mais comuns cometidos pelos alunos. Esta técnica é utilizada para dar uma resposta automática para o aluno, permitindo que ele receba uma avaliação mais rápida que a dada pelo professor. Segundo o estudo XXX, o uso de aprendizagem de máquina permitiu criar uma ferramenta para detectar erros de sintaxe corrigi-los. Um texto publicado no site da Microsoft, mostrou como o uso do algoritmo de \textit{Deep Learning} foi eficaz para identificar as correções necessárias que o aluno deveria fazer para que o programa estivesse correto (reescrever).

A pergunta que esta dissertação irá responder é quais erros são identificáveis e mais comuns que uma ferramenta de aprendizagem de máquina consegue detectar a partir de um repositório de código-fonte previamente analisados? Quais são os tipos de erros facilmente identificáveis? Qual a diferença entre os erros encontrados pela ferramenta e os erros identificados na literatura, principalmente, erros obtidos através de entrevistas com alunos e professores? Esta técnica é aplicável a todas as linguagens? Caso seja aplicável em todas as linguagens, quais erros são mais facilmente identificáveis?


\section{Considerações Preliminares}
\label{sec:consideracoes_preliminares}



 

%% ------------------------------------------------------------------------- %%
\section{Objetivos}
\label{sec:objetivo}

Texto texto texto texto texto texto texto texto texto texto texto texto texto
texto texto texto texto texto texto texto texto texto texto texto texto texto
texto texto texto texto texto texto.

%% ------------------------------------------------------------------------- %%
\section{Contribuições}
\label{sec:contribucoes}



%% ------------------------------------------------------------------------- %%
\section{Organização do Trabalho}
\label{sec:organizacao_trabalho}

No Capítulo~\ref{cap:conceitos}, apresentamos os conceitos ... Finalmente, no
Capítulo~\ref{cap:conclusoes} discutimos algumas conclusões obtidas neste
trabalho. Analisamos as vantagens e desvantagens do método proposto ... 

As sequências testadas no trabalho estão disponíveis no Apêndice \ref{ape:sequencias}.
