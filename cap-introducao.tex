%% ------------------------------------------------------------------------- %%
\chapter{Introdução}
\label{cap:introducao}

Em 1936, Alan Turing publicou um artigo chamado \textit{On Computable Numbers, with an application to the Entscheidungsproblem} (\citeyear{turing1936a}) no qual descreveu uma máquina teórica capaz de resolver qualquer problema que pudesse ser descrito através de instruções pré-definidas numa fita de papel. Este artigo tornaria-se a base para a computação e invenção dos computadores. Durante a segunda guerra, Turing construiu uma máquina capaz de quebrar a maioria do código Morse secreto usado nas comunicações pela força naval alemã.  Segundo alguns especialistas, a guerra teria durado mais tempo se não fosse a máquina inventada por Alan Turing para quebrar a criptografia das mensagens. A humanidade deu um grande salto graças ao avanço dos computadores desde o fim da segunda guerra até os dias de hoje. Este salto tecnológico permitiu o avanço em diversas áreas como saúde, telecomunicações, aviação, engenharia, financeira etc. 

Com o aumento do poder de processamento computacional dobrando a cada 18 meses, Lei de Moore, novos avanços em tecnologia surgem como a Internet da Coisas (IoT), cidades inteligentes (Smart Cities), Indústria 4.0. Todas tem um ponto em comum: computação úbiqua, uso de inteligência artificial e big data. E um outro ponto é a automação. Governos em todo o mundo estão preocupados com o advento da inteligência artificial e automação, pois muitos especialistas dizem que haverá um desemprego em massa. A maioria dos empregos que tem uma tarefa repetitiva, que não exige criatividade e interação social, serão automatizados. Em sua maioria, são empregos exercidos por pessoas com baixa qualificação \citep{europarl:2016}. Segundo um relatório da OCDE, este desemprego em massa não ocorrerá tão rapidamente, pois o computador consegue fazer somente tarefas bem definidas e a maioria dos trabalhos não tem uma definição clara das tarefas a serem feitas \citep{ArntzGregoryZierahn16oecd}.

Em um item todos os governos concordam: há uma necessidade em preparar a nova geração para esta nova era. Hoje há um defícit de programadores em todo o mundo. E com o advento do IoT, Indústria 4.0, a demanda por desenvolvedores e programadores irá aumentar ainda mais. Governos de todo o mundo estão empenhados em um projeto de ensino de programação no currículo básico das escolas para atender a esta demanda. Alan Perlis disse em 1960 que programação deveria ser ensinada a todos. O modo de pensar "algoritmicamente" é útil para diversos problemas em nosso cotidiano. E com a popularização e a presença em massa dos computadores em nosso cotidiano, a importância de saber programar aumenta. Governos e empresas multinacionais juntaram forças para ensinar programação à populacao através de iniciativas como o code.org, udacity, scratch etc.

Mas aprender a programar não é uma tarefa fácil. Diversos estudos que apontam para a alta taxa de evasão dos alunos dos cursos de Ciência da Computação \citep{Watson:2014:FRI:2591708.2591749}. E outros estudos apontam a alta taxa de reprovação por parte dos alunos nos cursos de introdução à programação \citep{Watson:2014:FRI:2591708.2591749, bosse:2015:cbie}. Segundo um estudo preliminar feito por \cite{bosse:2015:cbie}, a taxa de reprovação nos cursos introdutórios de programação ultrapassam a taxa de 30\% somente na Universidade de São Paulo. A reprovação no curso de introdução à programação é um dos fatores que contribuem para a evasão do curso. Segundo os dados do censo do Ensino Superior divulgado pelo MEC em 2016, a taxa de evasão média para cursos de bacheralado é em torno de 22\%, enquanto para os cursos de bacharelado em Computação, a taxa de evasão supera 28\%.

Segundo \cite{Kalelioglu:2016}, o pensamento computacional é um processo para resolução de problemas. Inicia-se com a identificação do problema através da abstração e decomposição. Para entender melhor e solucionar o problema, é necessário coletar, representar e analisar os dados. No processo de coleta e análise, reconhecimento de padrões, conceptualização e representação dos dados devem ser levados em consideração. Para obter soluções mais precisas, processos cognitivos como lógica matemática, algortimos podem ser empregados. Para implementar a solução, modelagem, automação e simulação deve ser utializada para verificar a eficácia da solução. Por último, aplicar a solução obtida para diferentes tipos de problemas deve ser realizado para generalizar a solução. Segundo \cite{Kalelioglu:2016} o pensamento computacional envolve diversos modelos mentais como abstração, resolução de problemas, reconhecimento de padrões, decomposição, generalização etc.

Além dos alunos não terem boa parte dos modelos mentais descritos por \cite{Kalelioglu:2016}, há diversos outros fatores que dificultam a aprendizagem de programação. Há fatores sociais, motivação, linguagem de programação escolhida, didática e método de ensino utilizada pelo professor \citep{bosse:2017}. E segundo o estudo YYY, os instrutores e professores tem dificuldade em identificar quais são os erros mais cometidos pelos alunos. De acordo com o YYY, a maioria dos professores não conseguiu identificar quais são as dificuldades dos alunos, quando confrontados com entrevistas feitas com ambos e após uma análise (reescrever).

Há estudos que tentam identificar quais são os erros mais comuns cometidos pelos alunos na aprendizagem de programação. Porém, a maioria dos estudos adotam a abordagem de entrevista com alunos e professores, catalogação dos erros verificados e posterior análise para validá-los (verificar se está certo??) \cite{Hristova:2003:ICJ:792548.611956}.  (dificuldade e erro sao a mesma coisa?)

Com o aumento do poder computacional, crescente uso de ferramentas de aprendizagem de máquina e a grande oferta de repositórios de código-fontes como Github, por exemplo, nasce o termo Big Code. Big Code refere-se ao conhecimento que podemos adquirir a partir da análise dos repositórios de código fonte através de ferramentas de aprendizagem de máquina. Este é um novo desafio, construir ferramentas que sejam capazes de aprender a partir dos repositórios de código fonte. Com o conhecimento que podemos adquirir a partir da análise dos repositórios, podemos construir ferramentas para analisar a arquitetura de um projeto, verificar se o código fonte ou o nome do método está de acordo com a convenção, detectar erros de execução de forma antecipada. (reescrever com referências)

Com as ferramentas de aprendizagem de máquina, é possível também identificar quais são os erros mais comuns cometidos pelos alunos. Esta técnica é utilizada para dar uma resposta automática para o aluno, permitindo que ele receba uma avaliação mais rápida que a dada pelo professor. Segundo o estudo XXX, o uso de aprendizagem de máquina permitiu criar uma ferramenta para detectar erros de sintaxe corrigi-los. Um texto publicado no site da Microsoft, mostrou como o uso do algoritmo de \textit{Deep Learning} foi eficaz para identificar as correções necessárias que o aluno deveria fazer para que o programa estivesse correto (reescrever).

A pergunta que esta dissertação irá responder é quais erros são identificáveis e mais comuns que uma ferramenta de aprendizagem de máquina consegue detectar a partir de um repositório de código-fonte previamente analisados? Quais são os tipos de erros facilmente identificáveis? Qual a diferença entre os erros encontrados pela ferramenta e os erros identificados na literatura, principalmente, erros obtidos através de entrevistas com alunos e professores? Esta técnica é aplicável a todas as linguagens? Caso seja aplicável em todas as linguagens, quais erros são mais facilmente identificáveis?


\section{Considerações Preliminares}
\label{sec:consideracoes_preliminares}

 

%% ------------------------------------------------------------------------- %%
\section{Objetivos}
\label{sec:objetivo}



%% ------------------------------------------------------------------------- %%
\section{Contribuições}
\label{sec:contribucoes}



%% ------------------------------------------------------------------------- %%
\section{Organização do Trabalho}
\label{sec:organizacao_trabalho}

No Capítulo~\ref{cap:conceitos}, apresentamos os conceitos ... Finalmente, no
Capítulo~\ref{cap:conclusoes} discutimos algumas conclusões obtidas neste
trabalho. Analisamos as vantagens e desvantagens do método proposto ... 

As sequências testadas no trabalho estão disponíveis no Apêndice \ref{ape:sequencias}.
