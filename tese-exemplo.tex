% Arquivo LaTeX de exemplo de dissertação/tese a ser apresentados ?CPG do IME-USP
% 
% Vers? 5: Sex Mar  9 18:05:40 BRT 2012
%
% Cria?o: Jes?s P. Mena-Chalco
% Revis?: Fabio Kon e Paulo Feofiloff
%  
% Obs: Leia previamente o texto do arquivo README.txt

\documentclass[11pt,twoside,a4paper]{book}

% ---------------------------------------------------------------------------- %
% Pacotes 
\usepackage{fontspec}
\usepackage{polyglossia}
\setdefaultlanguage{brazil}
%\usepackage[bf,small,compact]{titlesec} % cabe?lhos dos t?ulos: menores e compactos
%\usepackage[fixlanguage]{babelbib}

\usepackage[xetex]{graphicx}           % usamos arquivos pdf/png como figuras
\usepackage{setspace}                   % espa?mento flex?el
\usepackage{indentfirst}                % indenta?o do primeiro par?rafo
\usepackage{makeidx}                    % ?dice remissivo
\usepackage[nottoc]{tocbibind}          % acrescentamos a bibliografia/indice/conteudo no Table of Contents
\usepackage{courier}                    % usa o Adobe Courier no lugar de Computer Modern Typewriter
\usepackage{type1cm}                    % fontes realmente escal?eis
\usepackage{listings}                   % para formatar c?igo-fonte (ex. em Java)
\usepackage{titletoc}
\usepackage[font=small,format=plain,labelfont=bf,up,textfont=it,up]{caption}
\usepackage[usenames,svgnames,dvipsnames]{xcolor}
\usepackage[a4paper,top=2.54cm,bottom=2.0cm,left=2.0cm,right=2.54cm]{geometry} % margens
%\usepackage[pdftex,plainpages=false,pdfpagelabels,pagebackref,colorlinks=true,citecolor=black,linkcolor=black,urlcolor=black,filecolor=black,bookmarksopen=true]{hyperref} % links em preto
\usepackage[plainpages=false,pdfpagelabels,pagebackref,colorlinks=true,citecolor=DarkGreen,linkcolor=NavyBlue,urlcolor=DarkRed,filecolor=green,bookmarksopen=true]{hyperref} % links coloridos
\usepackage[all]{hypcap}                    % soluciona o problema com o hyperref e capitulos
\usepackage[round,sort,nonamebreak]{natbib} % cita?o bibliogr?ica textual(plainnat-ime.bst)
\fontsize{60}{62}\usefont{OT1}{cmr}{m}{n}{\selectfont}

% ---------------------------------------------------------------------------- %
% Cabe?lhos similares ao TAOCP de Donald E. Knuth
\usepackage{fancyhdr}
\pagestyle{fancy}
\fancyhf{}
\renewcommand{\chaptermark}[1]{\markboth{\MakeUppercase{#1}}{}}
\renewcommand{\sectionmark}[1]{\markright{\MakeUppercase{#1}}{}}
\renewcommand{\headrulewidth}{0pt}

% ---------------------------------------------------------------------------- %
\graphicspath{{./figuras/}}             % caminho das figuras (recomend?el)
\frenchspacing                          % arruma o espa?: id est (i.e.) e exempli gratia (e.g.) 
\urlstyle{same}                         % URL com o mesmo estilo do texto e n? mono-spaced
\makeindex                              % para o ?dice remissivo
\raggedbottom                           % para n? permitir espa?s extra no texto
\fontsize{60}{62}\usefont{OT1}{cmr}{m}{n}{\selectfont}
\cleardoublepage
\normalsize

% ---------------------------------------------------------------------------- %
% Op?es de listing usados para o c?igo fonte
% Ref: http://en.wikibooks.org/wiki/LaTeX/Packages/Listings
\lstset{ %
language=Java,                  % choose the language of the code
basicstyle=\footnotesize,       % the size of the fonts that are used for the code
numbers=left,                   % where to put the line-numbers
numberstyle=\footnotesize,      % the size of the fonts that are used for the line-numbers
stepnumber=1,                   % the step between two line-numbers. If it's 1 each line will be numbered
numbersep=5pt,                  % how far the line-numbers are from the code
showspaces=false,               % show spaces adding particular underscores
showstringspaces=false,         % underline spaces within strings
showtabs=false,                 % show tabs within strings adding particular underscores
frame=single,	                % adds a frame around the code
framerule=0.6pt,
tabsize=2,	                    % sets default tabsize to 2 spaces
captionpos=b,                   % sets the caption-position to bottom
breaklines=true,                % sets automatic line breaking
breakatwhitespace=false,        % sets if automatic breaks should only happen at whitespace
escapeinside={\%*}{*)},         % if you want to add a comment within your code
backgroundcolor=\color[rgb]{1.0,1.0,1.0}, % choose the background color.
rulecolor=\color[rgb]{0.8,0.8,0.8},
extendedchars=true,
xleftmargin=10pt,
xrightmargin=10pt,
framexleftmargin=10pt,
framexrightmargin=10pt
}

% ---------------------------------------------------------------------------- %
% Corpo do texto
\begin{document}
\frontmatter 
% cabe?lho para as p?inas das se?es anteriores ao cap?ulo 1 (frontmatter)
\fancyhead[RO]{{\footnotesize\rightmark}\hspace{2em}\thepage}
\setcounter{tocdepth}{2}
\fancyhead[LE]{\thepage\hspace{2em}\footnotesize{\leftmark}}
\fancyhead[RE,LO]{}
\fancyhead[RO]{{\footnotesize\rightmark}\hspace{2em}\thepage}

\onehalfspacing  % espa?mento

% ---------------------------------------------------------------------------- %
% CAPA
% Nota: O t?ulo para as disserta?es/teses do IME-USP devem caber em um 
% orif?io de 10,7cm de largura x 6,0cm de altura que h?na capa fornecida pela SPG.
\thispagestyle{empty}
\begin{center}
    \vspace*{2.3cm}
    \textbf{\Large{Uso de redes neurais recorrentes na detecção de padrões de erros cometidos por alunos novatos}}\\
    
    \vspace*{1.2cm}
    \Large{Marcelo de Rezende Martins}
    
    \vskip 2cm
    \textsc{
    Dissertação apresentada\\[-0.25cm] 
    ao\\[-0.25cm]
    Instituto de Pesquisas Tecnológicas do Estado de São Paulo - IPT\\[-0.25cm]
    para\\[-0.25cm]
    obtenção do título\\[-0.25cm]
    de\\[-0.25cm]
    Mestre em Engenharia da Computação}
    
    \vskip 1.5cm
    Área: Engenharia de Software\\
    Orientador: Prof. Dr. Marco Aurélio Gerosa

   	\vskip 1cm
    \normalsize{Durante o desenvolvimento deste trabalho o autor recebeu auxílio
    financeiro da FAPESP}
    
    \vskip 0.5cm
    \normalsize{\today}
\end{center}

% ---------------------------------------------------------------------------- %
% P?ina de rosto (S?PARA A VERS? DEPOSITADA - ANTES DA DEFESA)
% Resolu?o CoPGr 5890 (20/12/2010)
%
% IMPORTANTE:
%   Coloque um '%' em todas as linhas
%   desta p?ina antes de compilar a vers?
%   final, corrigida, do trabalho
%
%
\newpage
\thispagestyle{empty}
    \begin{center}
        \vspace*{2.3 cm}
        \textbf{\Large{Uso de redes neurais recorrentes na detecção de padrões de erros cometidos por alunos novatos}}\\
        \vspace*{2 cm}
    \end{center}

    \vskip 2cm

    \begin{flushright}
	Esta é a versão original da dissertação elaborada pelo\\
	candidato (Marcelo de Rezende Martins), tal como \\
	submetida à Comissão Julgadora.
    \end{flushright}

\pagebreak


% ---------------------------------------------------------------------------- %
% P?ina de rosto (S?PARA A VERS? CORRIGIDA - AP? DEFESA)
% Resolu?o CoPGr 5890 (20/12/2010)
%
% Nota: O t?ulo para as disserta?es/teses do IME-USP devem caber em um 
% orif?io de 10,7cm de largura x 6,0cm de altura que h?na capa fornecida pela SPG.
%
% IMPORTANTE:
%   Coloque um '%' em todas as linhas desta
%   p?ina antes de compilar a vers? do trabalho que ser?entregue
%   ?Comiss? Julgadora antes da defesa
%
%
\newpage
\thispagestyle{empty}
    \begin{center}
        \vspace*{2.3 cm}
        \textbf{\Large{Uso de redes neurais recorrentes na detecção de padrões de erros cometidos por alunos novatos}}\\
        \vspace*{2 cm}
    \end{center}

    \vskip 2cm

    \begin{flushright}
	Esta versão da dissertação/tese contém as correções e alterações sugeridas\\
	pela Comissão Julgadora durante a defesa da versão original do trabalho,\\
	realizada em \today. Uma cópia da versão original está disponível no\\
	Instituto de Matemática e Estatística da Universidade de São Paulo.

    \vskip 2cm

    \end{flushright}
    \vskip 4.2cm

    \begin{quote}
    \noindent Comissão Julgadora:
    
    \begin{itemize}
		\item Profa. Dr. Nome Completo (orientadora) - IPT [sem ponto final]
		\item Prof. Dr. Nome Completo - IPT [sem ponto final]
		\item Prof. Dr. Nome Completo - IPT [sem ponto final]
    \end{itemize}
      
    \end{quote}
\pagebreak


\pagenumbering{roman}     % come?mos a numerar 

% ---------------------------------------------------------------------------- %
% Agradecimentos:
% Se o candidato n? quer fazer agradecimentos, deve simplesmente eliminar esta p?ina 
\chapter*{Agradecimentos}
Texto texto texto texto texto texto texto texto texto texto texto texto texto
texto texto texto texto texto texto texto texto texto texto texto texto texto
texto texto texto texto texto texto texto texto texto texto texto texto texto
texto texto texto texto. Texto opcional.


% ---------------------------------------------------------------------------- %
% Resumo
\chapter*{Resumo}

\noindent Martins, M. R. \textbf{Uso de redes neurais recorrentes na detecção de padrões de erros cometidos por alunos novatos}. 
2018. 120 f.
Dissertação (Mestrado) - Instituto de Pesquisas Tecnológicas do Estado de São Paulo,
IPT, 2018.
\\

Elemento obrigatório, constituído de uma sequência de frases concisas e
objetivas, em forma de texto.  Deve apresentar os objetivos, métodos empregados,
resultados e conclusões.  O resumo deve ser redigido em parágrafo único, conter
no mínimo 500 palavras e ser seguido dos termos representativos do conteúdo do
trabalho (palavras-chave). 
Texto texto texto texto texto texto texto texto texto texto texto texto texto
texto texto texto texto texto texto texto texto texto texto texto texto texto
texto texto texto texto texto texto texto texto texto texto texto texto texto
texto texto texto texto texto texto texto texto texto texto texto texto texto
texto texto texto texto texto texto texto texto texto texto texto texto texto
texto texto texto texto texto texto texto texto.
Texto texto texto texto texto texto texto texto texto texto texto texto texto
texto texto texto texto texto texto texto texto texto texto texto texto texto
texto texto texto texto texto texto texto texto texto texto texto texto texto
texto texto texto texto texto texto texto texto texto texto texto texto texto
texto texto.
\\

\noindent \textbf{Palavras-chave:} padrões de erros, redes neurais recorrentes, aprendizagem de máquina, dificuldades de aprendizagem, inteligência artificial para educação.

% ---------------------------------------------------------------------------- %
% Abstract
\chapter*{Abstract}
\noindent MARTINS, M. R. \textbf{Using recurrent neural networks for detection of novice's most common errors}. 
2018. 120 f.
Dissertação (Mestrado) - Instituto de Pesquisas Tecnológicas do Estado de São Paulo,
IPT, São Paulo, 2018.
\\


Elemento obrigatório, elaborado com as mesmas características do resumo em
língua portuguesa. De acordo com o Regimento da Pós- Graduação da USP (Artigo
99), deve ser redigido em inglês para fins de divulgação. 
Text text text text text text text text text text text text text text text text
text text text text text text text text text text text text text text text text
text text text text text text text text text text text text text text text text
text text text text text text text text text text text text.
Text text text text text text text text text text text text text text text text
text text text text text text text text text text text text text text text text
text text text.
\\

\noindent \textbf{Keywords:} recurrent neural networks, machine learning, difficulties, errors, ai for education, computer science education, Patterns of problems, novice, casual developer, programming, software engineering education, Difficulties, barriers, programming learning,
novices, introduction to programming, computational thinking.

% ---------------------------------------------------------------------------- %
% Sum?io
\tableofcontents    % imprime o sum?io

% ---------------------------------------------------------------------------- %
\chapter{Lista de Abreviaturas}
\begin{tabular}{ll}
         IoT         & Internet das Coisas (\emph{Internet of Things})\\
\end{tabular}

% ---------------------------------------------------------------------------- %
\chapter{Lista de Símbolos}
\begin{tabular}{ll}
        $\omega$    & Frequência angular\\
        $\psi$      & Função de análise \emph{wavelet}\\
        $\Psi$      & Transformada de Fourier de $\psi$\\
\end{tabular}

% ---------------------------------------------------------------------------- %
% Listas de figuras e tabelas criadas automaticamente
\listoffigures            
\listoftables            

% ---------------------------------------------------------------------------- %
% Cap?ulos do trabalho
\mainmatter

% cabe?lho para as p?inas de todos os cap?ulos
\fancyhead[RE,LO]{\thesection}

\singlespacing              % espa?mento simples
%\onehalfspacing            % espa?mento um e meio

\input cap-introducao        % associado ao arquivo: 'cap-introducao.tex'
\input cap-conceitos         % associado ao arquivo: 'cap-conceitos.tex'
\input cap-conclusoes        % associado ao arquivo: 'cap-conclusoes.tex'

% cabe?lho para os ap?dices
\renewcommand{\chaptermark}[1]{\markboth{\MakeUppercase{\appendixname\ \thechapter}} {\MakeUppercase{#1}} }
\fancyhead[RE,LO]{}
\appendix

\chapter{Sequências}
\label{ape:sequencias}

Texto texto texto texto texto texto texto texto texto texto texto texto texto
texto texto texto texto texto texto texto texto texto texto texto texto texto
texto texto texto texto texto texto.


      % associado ao arquivo: 'ape-conjuntos.tex'

% ---------------------------------------------------------------------------- %
% Bibliografia
\backmatter \singlespacing   % espa?mento simples
\bibliographystyle{plainnat-ime} % cita?o bibliogr?ica textual
\bibliography{bibliografia}  % associado ao arquivo: 'bibliografia.bib'

% ---------------------------------------------------------------------------- %
% ?dice remissivo
\index{TBP|see{periodicidade região codificante}}
\index{DSP|see{processamento digital de sinais}}
\index{STFT|see{transformada de Fourier de tempo reduzido}}
\index{DFT|see{transformada discreta de Fourier}}
\index{Fourier!transformada|see{transformada de Fourier}}

\printindex   % imprime o ?dice remissivo no documento 

\end{document}
